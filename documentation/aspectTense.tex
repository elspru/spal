\section{Tense}
\label{tense}
\begin{description}
    \item[past-tense]
    things that happened
    \item[hesternal-tense]
    yesterday
    \item[recent-past-tense]
    
    \item[remote-past-tense]
    
    \item[present-tense]
    now, things that are happening
    \item[hodiernal-tense]
    today
    \item[future-tense]
    things that will happen
    \item[crastinal-tense]
    tommorrow
    \item[soon-future-tense]
    
    \item[remote-future-tense]
    
\end{description}
\section{Aspects}
\begin{multicols}{2}
\begin{description}
\item[atelic-aspect] a \\
 cumulative-reference process 
for SPEL it is like signal processing,
at any point it is still processing, perhaps for parallel processes
\item[cessative-aspect]
for ending process \\
for SPEL exiting process \\
for hardware description language falling edge
\item[completive-aspect]
completely and thoroughly finished \\
for SPEL finished with no errors
\item[continuative-aspect]
process started but not active\\
for SPEL idle processes
\item[delimitative-aspect]
temporary process
\item[frequentive-aspect]
repetitive process\\
for SPEL can be used for servers/daemons
\item[gnomic-aspect]
general truths\\
for SPEL defining functions
\item[habitual-aspect]
habitual process \\
for SPEL can be provided services or features
\item[inchoative-aspect]
begining of process \\
for SPEL loading process \\
for Hardware Description Layer signal rising edge
\item[imperfective-aspect] 
for SPEL a process which is ongoing \\
any partial process
\item[momentane-aspect]
for things that happen suddenly or momentarily,
like power surges and lightning bolts. For instance a clock-tick could be momentane and frequentive.

\item[perfective-aspect]
any whole process \\
for SPEL a process which has completed

\item[progressive-aspect]
for active process\\
for SPEL for active processes
\item[prospective-aspect]
for processes that happen after \\
for SPEL queued processes
\item[retrospective-aspect]
for processes that happen before \\
for SPEL prerequisite processes
%<!--\item[semelfactive-aspect]
%for something that happened once-->
\item[telic-aspect]
quantized process \\
for processes where any of the parts are not the whole,
only taken together is it the whole. \\
for SPEL this is processes that require sequential components of
a different kind.

\end{description}
\end{multicols}
\begin{table}
\caption{Aspect Tree}
\begin{itemize}
\item state
\begin{itemize}
    \item perfective-aspect
        \begin{itemize}
            \item momentane-aspect
            \item completive-aspect
            %<!--\item semelfactive-aspect-->
        \end{itemize}
    \item imperfective-aspect
        \begin{itemize}
            \item continuous-aspect
            \item progressive-aspect
            \item delimitative-aspect
            %<!--\item frequentive-aspect-->
        \end{itemize}
    
\end{itemize}

\item occurence
\begin{itemize}
    \item gnomic-aspect
    \item habitual-aspect
\end{itemize}

\item part of time
\begin{itemize}
    \item inchoative-aspect
    \item cessative-aspect
\end{itemize}

\item relative time
\begin{itemize}
    \item retrospective-aspect
    \item prospective-aspect
\end{itemize}

\end{itemize}
<h3>Lexical</h3>
\begin{itemize}
\item composition
    \begin{itemize}
        \item atelic
        \item telic
        \item stative-verb
    \end{itemize}

\item causal
     \begin{itemize}
     \item autocausative-verb
     \item anticausative-verb
     \end{itemize}
\end{itemize}
\label{aspecttree}
\end{table}
