%\documentclass[12pt]{report}
%\usepackage{machine_programmer}
%\usepackage{amstext}
%\begin{document}
\chapter{Pyash Encoding}

The virtual machine uses variable-length-instruction-word (VLIW),
loosely inspired by
\href{http://scale.eecs.berkeley.edu/papers/hat-cases2001.pdf}{head and
tails instruction format} (HTF). HTF uses VLIW's which are 128 or 256
bits long, however there can be multiple instructions per major
instruction word.

\section{VLIW's Head Index}\label{vliws-head-index}

The head is really a parse index, to show the phrase boundaries. In
TroshLyash each bit represents a word, each of which is 16bits, when a
phrase boundary is met then the bits flip from 1 to 0 or vice-versa at
the phrase boundary word.\ index takes up the first 16bits of the VLIW\@.
This would lead to 256bit (32Byte) VLIW's. The real advantage of the
indexing occurs when there either multiple sentences per VLIW, or when
there are complex sentences in the VLIW's. Having the VLIW's broken up
into 32Byte chunks, makes it easier to address function entry points,
which can be placed at the beginning of a VLIW\@. Can fit 16 VLIWS in a
POSIX page, 128 VLIW's in a Linux page, so would only need 1 byte
(8bits) for addressing functions that are within 1 page distance.

\section{Word Compression}\label{word-compression}

Now for the slightly more interesting issue of packing as many as 5
glyphs into a mere 16 bits. Why this is particularly interesting is that
there is an alphabet of 32 glyphs, which would typically required 5 bits
each, and thus 25bits in total. However the 16 bit compression is mostly
possible due to the rather strict phonotactics of TroshLyash, as only
certain classes of letters can occur in any exact place. The encoding
supports 4 kinds of words, 2 grammar word classes and 2 root word
classes. Where C is a consonant, T is a tone and V is a vowel, they are
CVT, CCVT, and CVTC, CCVTC respectively.

\subsection{CCVTC or CSVTF}\label{ccvtc-or-csvtf}

I'll start with explaining the simplest case of the CCVTC word pattern.
To make it easier to understand the word classes can call is the CSVTF
pattern, where S stands for Second consonant, and F stands for Final
Consonant. The first C represents 22 consonants, so there needs to be at
least 5 bits to represent them. Here are the various classes
\begin{description}
\item[``C'']:``p'',``t'',``k'',``f'', ``s'',``c'',``x'',
``b'',``d'',``g'',``v'', ``z'',``j'', ``n'',``m'',``q'',``r'',
``l'',``y'',``w'', 
\item[``S'']``f'',``s'',``c'',``y'', ``r'',``w'',``l'',``x'', ``z'',``j'',``v'',
\item[``V'']``i'',``a'',``e'',``o'',``u'',``6'',
\item[``T'']``7'',``\_'', 
\item[``F'']``p'',``t'',``k'',``f'',
``s'',``c'',``n'',``m''
\end{description}
, (can check the
\href{http://wyn.bot.nu/spel/src/vocab/gen/phonology.html}{phonology}
page for pronunciation) C needs 5 bits, S would need 4 bits, however the
phonotactics means that if the initial C is voiced, then the S must be
voiced, thus ``c'' would turn into ``j'', ``s'' into ``z'' and ``f''
into ``v'', also none of the ambigiously voiced phonemes (l, m, n, q, y,
w, r) can come before a fricative because they have a higher sonority,
thus must be closer to the vowel. So S only needs 3 bits. V needs 3 bits
T needs 2 bits and F needs 3 bits which is a total of 16 bits. 5+3+3+2+3
= 16 However there are other kinds of words also.\ we'll see how those
work.

\subsection{HCVTF}\label{hcvtf}

So here we have to realize that CVC or CVTC is actually HCVTF due to
alignment. So what we do is make a three bit trigger from the first
word, the trigger is 0, which can be three binary 0's, 0b000 3+5+3+2+3 =
16 H+C+V+T+C this does mean that now 0b1000, 0b10000 and 0b11000 is no
longer useable consonant representation, however since there are only 22
consonants, and only 2 of those are purely for syntax so aren't
necessary, so that's okay, simply can skip the assignment of 8, 16 and
24.

\subsection{CSVT}\label{csvt}

This is similar to the above, except we use 0b111 as the trigger,
meaning have to also skip assignment of 15, 23 and 31.
$3+5+3+3+2=16$?+C+S+V+T

\subsection{CVT}\label{cvt}

For this one can actually simply have a special number, such as 30,
which indicates that the word represents a 2 letter word. $5+5+3+2+1$
$F+C+V+T+P$ what is PF P can be a parity-bit for the phrase, or simply
unassigned.

\section{Quotes}\label{quotes}

Now with VM encodings, it is also necessary to make reference to binary
numbers and things like that. The nice thing with this encoding is that
we can represent several different things. Currently with the above
words, we have 1 number undefined in the initial 5 bits. 29 can be an
initial dot or the final one, can call the the quote-denote (QD),
depending on if parser works forwards or backwards. Though for
consistency it is best that it is kept as a suffix (final one), as most
other things are suffixes. 5+3+8 = 16 Q+L+B QD has a 3 bit argument of
Length. The Length is the number of 16bit fields which are quoted, if
the length is 0, then the B is used as a raw byte of binary. Otherwise
the B represents the encoding of the quoted bytes, mostly so that it is
easier to display when debugging. The type information is external to
the quotes themselves, being expressed via the available TroshLyash
words. So in theory it would be possible to have a number that is
encoded in UTF-8, or a string that is encoded as a
floating-point-number. Though if the VM interpreter is smart then it
will make sure the encoding is compatible with the type Lyash type, and
throw an error otherwise.

\section{Extension}\label{extension}

This encoding already can represent over 17,000 words, which if they
were all assigned would take 15bits, so it is a fairly efficient
encoding. However the amount of words can be extended by increasing
number of vowels, as well as tones. And it may even be possible to add
an initial consonant if only one or two of the quote types is necessary.
However this extension isn't likely to be necessary anytime in the near
future, because adult vocabulary goes up to around 17,000 words, which
includes a large number of synonyms. For instance the Lyash core words
were generated by combining several different word-lists, which were all
meant to be orthogonal, yet it turns out about half were internationally
synonyms, so were cut down from around eight thousand to around four
thousand words. It will be possible to flesh out the vocabulary with
compound words and more technical words later on. Also it might make
sense to supplant or remove some words like proper-names of countries.

\section{Encoding Tidbit Overview}
\begin{bytefield}[endianness=little, bitwidth=0.0625\linewidth]{16}
  \bitheader{0,2,4,6,8,10,12,14,16}\\
  \bitbox{5}{C} & \bitbox{3}{S} & \bitbox{3}{V} & \bitbox{2}{T} & \bitbox{3}{F}  \\
  \bitbox{3}{SRD} & \bitbox{5}{C} & \bitbox{3}{V} & \bitbox{2}{T} & \bitbox{3}{F}  \\
  \bitbox{3}{LGD} & \bitbox{5}{C} & \bitbox{3}{S} &\bitbox{3}{V} & \bitbox{2}{T}  \\
  \bitbox{5}{SGD} & \bitbox{5}{C} & \bitbox{3}{V} & \bitbox{2}{T} &
\bitbox{1}{P}  \\
  \bitbox{5}{QD} & \bitbox{11}{QS} \\
\end{bytefield}
%\caption{Encoding Table Brief}
%\label{table:main}
%\caption*{
\begin{tabular}{l l}
Legend
  C & Initial Consonant\\
  S & Secondary Consonant\\
  V & Vowel\\
  T & Tone\\
  F & Final Consonant\\
  SRD & Short Root Denote\\
  LGD & Long Grammar Denote\\
  SGD & Short Grammar Denote\\
  P & (optional) Phrase Parity Check tidbit\\
  QD & Quote Denote\\
  QS & Quote Sort~\ref{quotesort} \\
\end{tabular}
%}


\section{Table of Values}\label{values}


\begin{tabular}[c]{@{}llllll@{}}
\toprule
\#     & C   & S   & V & T   & F \\
width & 5   & 3   & 3 & 2   & 3 \\
\midrule
0     & SRD   & y   /j/   & i /i/   & MT \textipa{/\tone{33}/}/  & m /m/ \\
1     & m /m/ & w   /w/   & a /\textipa{\"a}/ & 7  /\textipa{\tone{44}}/  & k /k/ \\
2     & k /k/ & s z /s z/ & u /u/ & \_ /\textipa{\tone{22}}/  & p /p/ \\
3     & y /j/ & l   /l/   & e /e̞/ &  5  /\textipa{\tone{42}}/ (U) & n /n/ \\
4     & p /p/ & f v /f v/ & o /o̞/ &         & s /s/ \\
5     & w /w/ & c j /\textipa{S Z}/ & 6 /\textipa{@}/ &         & t /t/ \\
6     & n /n/  & r /r/    &  4 /\textipa{1}/ (U)&         & f /f/ \\
7     & LGD   & x /\textipa{x G}/   & 3 /\textipa{\ae}/ (U) &         & c /\textipa{S}/ \\
8     & SRO   &           &       &         &   \\
9     & s /s/ &           &       &         &   \\
10    & t /t/ &           &       &         &   \\
11    & l /l/ &           &       &         &   \\
12    & f /f/ &           &       &         &   \\
13    & c /\textipa{S}/ &           &       &         &   \\
14    & r /r/ &           &       &         &   \\
15    & LGO   &           &       &         &   \\
16    & SRO   &           &       &         &   \\
17    & b /b/ &           &       &         &   \\
18    & g /g/ &           &       &         &   \\
19    & d /d/ &           &       &         &   \\
20    & z /z/ &           &       &         &   \\
21    & j /\textipa{Z}/ &           &       &         &   \\
22    & v /v/ &           &       &         &   \\
23    & LGO   &           &       &         &   \\
24    & SRO   &           &       &         &   \\
25    & q /\textipa{N}/ &           &       &         &   \\
26    & x /\textipa{G}/ &           &       &         &   \\
27    & 1 /\textipa{|}/ &           &       &         &   \\
28    & 8 /\textipa{||}/ &           &       &         &   \\
29    & QD    &           &       &         &   \\
30    & SGD   &           &       &         &   \\
31    & LGO   &           &       &         &   \\
\bottomrule
\end{tabular}
%\caption{encoding specification}
%\label{table:specification}
%\caption*{
\begin{tabular}{ll}
 & blank means out of bounds\\
U & unused\\
MT & middle tone, no marking\\
QD & quote denote\\
SGD & short grammar word denote\\
SRD & short root word denote\\
LGD & long grammar word denote\\
SRO & short root word denote overflow\\
LGO & long grammar word denote overflow\\
\end{tabular}
%}


\section{Quote Sort}
\label{quotesort}
\begin{bytefield}[endianness=little, bitwidth=0.0625\linewidth]{16}
  \bitheader{0,5,6,8,11,13,15}\\
  \bitbox{5}{} & \bitbox{11}{QS} \\
  \bitbox{5}{QD} & \bitbox{1}{NL} & \bitbox{2}{R} & \bitbox{3}{VT} & 
  \bitbox{2}{ST}  & \bitbox{3}{SD}  \\
\end{bytefield}

\subsection{definitions}
\begin{description}
  \item [QS] quote sort 
  \item [QD] quote denote
  \item [NL] name or literal bit
  \item [R] region
  \item [VT] vector thick
  \item [ST] scalar thick
  \item [SD] sort denote
\end{description}

\begin{tabular}{*{6}{l}}
\toprule
\multicolumn{6}{|c|}{16 tidbit}\\
\midrule
5 tidbit     & 1 tidbit & 2 tidbit            & 3 tidbit        & 2 tidbit & 3 tidbit \\
\midrule
quote denote & literal or name & region & vector thick & scalar thick & sort denote \\
\midrule
\midrule
\multicolumn{6}{|c|}{definitions}\\
\bottomrule
0            & literal& private  & 1  & 1 byte, 8 tidbit      & letter (s) \\
1            & name   & world    & 2 & 2 byte, 16 tidbit     & word (s) \\
2            &        & eternal  & 4 & 4 byte, 32 tidbit     & sentence (s) \\
3            &        & coworker & 8     & 8 byte, 64 tidbit     & database \\
4            &         & & 16                                  & & unsigned integer \\
5            &         & & U                                   & & signed integer \\
6            &         & & U                                   & & floating point number \\
7            &         & & 3                                   & & function \\
\bottomrule
\end{tabular}
The quote denote is 5 bits long, leaving 11 bits.\
the next 2 bits is used to indicate bit thickness of quote scalar (s),
the following 3 bits is used to indicate the magnitude of the vector (s),
1 bit for name or literal
%\begin{enumerate}
%  \setcounter{enumi}{0} 
%  \item literal
%  \item name
%\end{enumerate}
%
%2 bit for region
%\begin{enumerate}
%  \setcounter{enumi}{0} 
%  \item private memory
%  \item world memory 
%  \item eternal memory 
%  \item coworker memory 
%\end{enumerate}
%3 bits for noun denote:
%\begin{enumerate}
%  \setcounter{enumi}{0}
%  \item letter
%  \item word
%  \item sentence
%  \item text
%  \item unsigned integer
%  \item signed integer
%  \item floating point number
%  \item function
%\end{enumerate}
%\section{Denote Syntax}
%For literals
%\begin{description}
%  \item[letter] l \_letter
%  \item[word] word \_word
%  \item[phrase] word \_acc \_phrase
%  \item[sentence] word \_acc \_rea \_independent\_clause
%  \item[text] 
%  \item[function]
%  \item[datastructure]
%  \item[named data type]
%  \item[unsigned integer] one two three \_number (291)
%  \item[signed integer] one two three \_negatory\_quantifier \_num
%(-291)
%  \item[floating\_point\_number] two four \_floating\_point\_num ten \_bas
%one \_neg \_exponential \_num (2.4)
%  \item[fixed\_point\_number] two \_flo one \_num (2.1)
%  \item[rational] one \_rational three \_num (1/3)
%  \item[decimal number] ten \_bas  one one \_num (11)
%  \item[hexadecimal number] sixteen \_bas eleven \_num (11)
%  \item[vector] world \_word \_and \_voc \_word two sixteen word \_vector (vector of 16
%unsigned shorts each short containing a word, intialized to repeating sequence
%of ``hello \_vocative\_case'')
%\end{description}

In the case of a refferential, or variable name, the name can be (up to) four
words long, that way it fits in a 64bit area --- similar to a 64bit address.

\section{Independent-Clause Code Name}

Decided to make the independant-clause code name actually a universal hash,
based on the sorts, cases, aspects and mood of the sentence. It's easier that
way. 


%
%Each independent-clause can have a code name to help find it's program.
%
%There is a mix of grammatical-cases, sorts and a verb in each independent clause. 
%For matching with modern computer processing, a 64bit thickness is desired,
%though a 128 bit thickness may be possible.\ 
%
The grammatical cases can have a table to make it easy to identify them. 

\begin{table}
\begin{tabular}{l l l l l l}
                        & 0 source-case   & 1 way-case       & 2 destination-case   & 3 location-case   \\
0 base                  & nominative-case & instrumental-case& dative-case          & accusative-case   \\
1 space-context (x)     & ablative-case   & prosecutive-case & allative-case        & locative-case     \\
2 genitive-case         & possessive-case & descriptive-case & possessed-case       & relational-case   \\
3 discourse-context     & initiative-case & topic-case       & terminative-case     & vocative-case     \\
4 social-context        & causal-case     & evidential-case  & benefactive-case     & comitative-case   \\
5 surface-context (y)   & delative-case   & vialis-case      & superlative-case     & superessive-case  \\
6 interior-context (z)  & elative-case    & perlative-case   & illative-case        & inessive-case     \\
7 time-context (t)      & initial-time    & during-time      & final-time           & temporal-case     \\
\end{tabular}
\caption{grammtical-case number system}
\end{table}
%
%five bits to designate the case, and 11 bits for the quote type. 
%
%The context will henceforward be referred to as scene, and the other half being
%the posture. 
%
%\medskip
%
%\begin{table}
%\begin{bytefield}[endianness=little, bitwidth=0.0625\linewidth]{16}
%  \bitheader{0,11,14,15}\\
%   \bitbox{11}{11} & \bitbox{3}{3} & \bitbox{2}{2} & \\
%   \bitbox{11}{QS} & \bitbox{3}{S} & \bitbox{2}{P} & \\
%\end{bytefield}
%\caption{grammtical-case code}
%\begin{description}
%  \item [QS] quote sort~\ref{quotesort}
%  \item [S] scene
%  \item [P] posture
%\end{description}
%\end{table}
%
%This does mean that any independentClause would have a maximum of 
%three grammatical-cases plus a verb if in 64 bit.
%
%\begin{table}
%\begin{bytefield}[endianness=little, bitwidth=0.125\linewidth]{8}
%  \bitheader{0,2,4,6}\\
%  \bitbox{8}{64 tidbit} \\
%  \bitbox{2}{G} & \bitbox{2}{G} & \bitbox{2}{G} & \bitbox{2}{V}  \\
%\end{bytefield}
%
%
%\begin{bytefield}[endianness=little, bitwidth=0.125\linewidth]{8}
%  \bitheader{0,2,4,6}\\
%  \bitbox{8}{VUL2 vector of two 64 tidbit numbers for searching} \\
%  \bitbox{8}{or VUS8 vector of eight 16 tidbit numbers for establishing} \\
%  \bitbox{2}{G} & \bitbox{2}{G} & \bitbox{2}{G} & \bitbox{2}{G}  \\
%  \bitbox{2}{V} & \bitbox{2}{V xor T} & \bitbox{2}{V xor A} & \bitbox{2}{Pe}  \\
%\end{bytefield}
%
%\caption{code name sketch}
%\begin{description}
%  \item [G] grammatical-case
%  \item [V] verb
%  \item [T] tense
%  \item [A] aspect
%  \item [Pe] perspective (grammatical-mood)
%\end{description}
%\end{table}
%
%Can also have seperate identifiers for the verb and the grammatical-cases,
%then it would be easier to have multi-word verbs.
%
%

%\end{document}
