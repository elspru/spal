\chapter{Specification}
SPAL simple compile to OpenCL.\@

\begin{sidewaystable}
\begin{tabulary}{\textwidth}{L|L|L|L}
Pyash & SPAL & C & file \\
\midrule 
\midrule 
kratta krathnimna li &
cardinal  \_top cardinal name \_nom \_rea & int main () \{ & 
cardinal\_name.c \\
\midrule
swicta hnimna li &
social \_top name \_nom \_rea & void name () \{ & cardinal\_name.c \\
\midrule
hmasta hnimna li&
mind \_top name \_nom \_rea & inline void name (); &
library\_cardinal\_name.h \\
& & inline void name\ () \{ & library\_cardinal\_name.c \\
\midrule
krathmasta hnimna li &
cardinal mind \_top name \_nom \_rea & kernel void name () \{ &
cardinal\_name.cl \\
\end{tabulary}
\begin{tabulary}{\textwidth}{L|L|L}
\toprule
htipdoyu txikka hciccu &
ten \_num \_ins indexFinger \_acc down \_con & \texttt{if (i < 0xA) \{} \\
\midrule 
zrundofi  & 
0 \_num \_return & return 0; \\
\midrule
fe & 
\_finally & \} \\
\midrule
hnimna tyindo cyah &
name \_nom three \_num \_cop & \texttt{name = 3;} \\
\midrule 
txikna zrondo cyah &
indexFinger \_nom zero \_num \_cop & \texttt{i = 0;} \\
\midrule 
htipdoyu txikka hciccu hyikdoyu plosliwa htekhromli&
ten \_num \_ins indexFinger \_acc down \_con 
indexFinger \_acc one \_num \_ins plus \_rea \_and library program \_rea & 
\texttt{for \{;i < 0xA;\@ ++i\}\{ library\_program ();\}}  \\
\end{tabulary}
\end{sidewaystable}


\section{stages of compilation}

\begin{enumerate}
  \item natural language text (perhaps)
  \item analytic language text
  \item SPAL language text
  \item SPAL encoded tiles
  \item (OpenCL) C with SPAL names
  \item (OpenCL) C with natural names (perhaps)
\end{enumerate}

\section{Method for implementation}
In theory can use any language for implementation.  
Though ideally would be a version of C which is similar to the above, 
so it could then be recoded in SPAL.\@

\section{answer verification}

The agree debug library is OpenCL and holy ceremeny (pure function) compatible. 

Ideally would have a way of listing many inputs and their corresponding outputs. 
If this could be fed to an OpenCL kernel that would be delicious. 

The agree debug library can be the ``testing framework'' for SPAL programs. 
So each agree statement adds a line to the newspaper,
after the program is complete it can list the statements in the newspaper,
saying those are the tests that failed. 
Additionally could have a list of the number that have passed.

I'm thinking can save both the line number, and the amount that have passed in
the first line of the newspaper. It can be an actual sentence, with two 16bit
spaces for the values.\ 

gzat na hnuc do lweh hnuc do mwah slak fa li

A newspaper until number with number succeeded.

A newspaper should be at least 16 sentences long, which is one page or 512
bytes, and less than or equal to 512 sentences, (32 pages), since that is the
most that could  fit in L1 memory with other processes. 


\section{Memory}

There is no dynamic allocation of memory, only static, until further notice.

This is because historically dynamic allocation of memory has led to many memory
leaks and other problems. 

\section{Control Flow}
Instead of the traditional for loop that relies on variables defined outside
it's bounds, the for loops in SPAL only contain a function, and a listener to
see if it should break early. 

That way based on the size of the for loop, the compiler can assign it to run on
single thread, multi thread CPU, or GPU.\@ 

\section{translate all independentClauses to C}

Any independent-clause can be turned into C.

can be of the form:
\begin{lstlisting}
sort1_case1_sort2_case2_verb_mood (sort1 name, sort2 name); 
\end{lstlisting}

\subsection{C Name Composition}

For C, will need to include the types of the names in order to properly  call
functions, otherwise would have to have extra searching to locate which
function is being refered to. 

This will make it a bit like Navajo or Swahili, where the noun class will be
mandatory. So we should have easy grammar words for them,

for names of things:
\begin{description}
  \item[plu] paucal-number 8bit 
  \item[do]  number 16bit
  \item[pu] plural-number 32bit
  \item[ml6h] multal-number 64bit
  \item[ml6hhsosve] multal-number sixteen vector, vector of 16 64bit values.
  \item[fe] referrential, pointer
  \item[crih] letter, char
  \item[crihfe] letter referrential, char *
\end{description}

It seems I would only need a hash table lookup for operating on the GPU,
seems like most of the other stuff can be done with a few conditionals. 

