\chapter{Operation Template}
\section{overview}
\subsection{translation}
An English programmer writes English text. 

An English encoder encodes the English text to the Pyash medium code. 

A Chinese translator decodes the Pyash medium code into Chinese
text.

A Chinese programmer writes Chinese text. 

A Chinese encoder encodes the Chinese text into the Pyash medium code. 

\subsection{Compiler}
A code compiler from the medium code, to a cardinal ``.c''
file, library header file, and library ``.c'' files, as well as a kernel ``.cl''
file, and library file of intermediate code.  

Clang compiler takes main and library ``.c'' files, and library header files,
and produces an byin binary.

The byin binary operator sets up the constant stack, input data and makes
writeable output data.

The host code starts the virtual machine kernel, and library kernels. 


\begin{figure}
\begin{tikzpicture}[node distance=0.5cm and 1cm,>=stealth',bend angle=45,auto]
    \node [transitionV,label=above:glisprom] (prom) {};
    \node [place,label=above:glishrom] (hrom) [right= of prom] {}
        edge[post]   (prom)
        edge[pre]   (prom);
    \node [transitionV,label=above:kfin] (kfin) [right= of hrom] {}
        edge[pre]   (hrom);
    \node [place,label=above:hkom] (hkom) [below right= of kfin] {}
        edge[pre]   (kfin);
    \node [transitionV,label=above:nyot] (nyot) [below left= of hkom] {}
        edge[post]   (hrom)
        edge[pre]   (hkom);
    \node [place,label=above:tw6nhrom] (tw6nhrom) [left= of nyot] {}
        edge[post]   (kfin)
        edge[pre]   (nyot);
    \node [transitionV,label=above:tw6nprom] (tw6nprom) [left= of tw6nhrom]{}
        edge[post] (tw6nhrom)
        edge[pre] (tw6nhrom);
    \node [transitionV,label=above:kl6n] (kl6n) [right= of hkom]{}
        edge[pre] (hkom);
    \node [place,label=above left:pren] (pren) [below= of kl6n] {}
        edge[pre]   (kl6n);
    \node [place,label=above:mlep] (mlep) [right= of kl6n] {}
        edge[pre]   (kl6n);
    \node [place,label=above:krat] (krat) [above = of mlep] {}
        edge[pre]   (kl6n);
    \node [place,label=above:htek] (htek) [below= of mlep] {}
        edge[pre]   (kl6n);
    \node [transitionV,label=above:clang] (clang) [right= of mlep]{}
        edge[pre] (mlep)
        edge[pre] (krat)
        edge[pre] (htek);
    \node [place,label=right:byin] (byin) [below= of clang] {}
        edge[pre]   (clang);
    \node [transitionV,label=above:hvim] (hvim) [below= of htek]{}
        edge[pre] (byin)
        edge[pre,out=160,in=-90,looseness=1] (hkom)
        edge[pre] (pren);
    \node [place,label=left:prentsin] (tsin) [left= of hvim] {}
        edge[pre] (hvim);
    \node [transitionV,label=left:prennrek] (nrek) [below= of tsin]{}
        edge[pre] (tsin);
    \node [place,label=right:swutsric] (swut) [right= of hvim] {}
        edge[pre] (hvim);
    \node [transitionV,label=right:prenhvic] (hvic) [below= of swut]{}
        edge[pre] (swut);
    \node [place,label=below:psas] (psas) [below left= of hvic] {}
        edge[pre] (hvic)
        edge[pre] (hvim)
        edge[pre] (nrek)
        edge[post,in=-90,out=180,looseness=1]   (nyot);
    \node [transitionV,label=above:frenhbuc] (hbuc) [below= of tw6nprom]{};
    \node [place,label=above:nrup] (nrup) [right= of hbuc] {}
        edge[post,looseness=1] (kfin)
        edge[pre] (hbuc);
    \node [place,label=below:frenhtet] (frenhtet) [below= of nrup] {}
        edge[pre] (nyot)
        edge[post] (hbuc);
  %  \node [place,label=above:$D$] (d) [below= of hrom] {};
  %  \node [transitionV,label=above:$W_n$] (wn) [right= of d] {}
  %      edge[pre]   (htet)
  %      edge[pre]   (d);
  %  \node [place,tokens=1,label=above:$I$] (p2) [above right=of kfin] {}
  %      edge[pre,out=-135,in=30,looseness=1,overlay]   (kfin)
  %      edge[post,out=180,in=60,looseness=1,overlay]   (kfin);
  %  \node [place,tokens=2,label=above:$P_3$] (p3) [below right=of kfin] {}
  %      edge[pre]   (kfin);
  %  \node [transitionV,label=above:$T_2$] (t2) [above right=of p3] {}
  %      edge[pre]   (p2)
  %      edge[pre]   (p3)
  %      edge[post,out=-110,in=-50,looseness=2,overlay]  (htet);
  %  \node [place,tokens=1, label=above:$P_4$] (p4) [above right=of t2] {}
  %      edge[pre]   (t2);
\end{tikzpicture}
\caption{Memory Template}
\label{memoryTemplate}
\begin{tabular}{ll}
  glisprom & English programmer \\
  twynprom & Chinese programmer \\
  glishrom & English program \\
  tw6nhrom & Chinese program \\
  frenhbuc & French user \\
  frenhtet & French text \\
  nrup & input \\
  kfin & encoder \\
  hkom & code \\
  nyot & translator \\
  kl6n & compiler \\
  psas & produce or output \\
  krat & cardinal file (main.c) \\
  mlep & template file (lib.h) \\
  htek & library file (lib.c) \\
  pren & parallel file (kernels.cl) \\
  hvim & runtime \\
  prentsin & parallel knowledge (parallel data) \\
  prennrek & parallel workers \\
  swutsric & holy ceremonies \\
  prenhvic & parallel virtual-machine \\
  psas & produce  (output) \\
  %C & Code in constant memory \\
  %D & input Data in constant memory \\
  %V & Variable index in global memory \\
  %W & Worker \\
  %I & Intermediate values processed in local memory\\
  %P & Produced data in global memory \\
\end{tabular}
\end{figure}
